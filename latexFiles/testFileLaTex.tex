\documentclass[11pt]{article}
\usepackage{graphicx} %Required for diagrams
\usepackage[bookmarks=true]{hyperref}
\usepackage{bookmark}%Required to do pdf bookmarking
\usepackage{hyperref}%Required for referencing website pages

\begin{document}
\begin{titlepage}
\begin{center}
\includegraphics[width=350px]{University_of_Pretoria_Logo.png}\newline
% Title
\textsc{\LARGE COS301 Mini Project Functional Requirements Specification}\newline


%\begin{minipage}{0.4\textwidth}
\textbf{Group 4B} \\
\begin{flushright} \large
Kyhle Ohlinger \emph{u11131952} \newline
Andrew Parkes \emph{u12189139} \newline
Sifiso Shabangu \emph{u12081622} \newline
Maret Stoffberg \emph{u11071762} \newline
New Member \emph{uxxxxxxxx} \newline
New Member \emph{uxxxxxxxx} \newline
New Member \emph{uxxxxxxxx} \newline \newline \newline
\end{flushright}
%\end{minipage}
Here's a link to \href{https://github.com/KyhleOhlinger/COS301-Group-4_B.git}{Github}.


\vfill

{\large Version 1}
\\
{\large \today}

\end{center}
\end{titlepage}


\tableofcontents	%Creates Table of contents from sections and subsections, etc...
\newpage


\section{Introduction}

The purpose of this document is to fully specify and outline the functional requirements of "The use of Online Discussions in Teaching (TODT)" research project, received from the Computer Science Education Didactic and Applications Research (CSEDAR) team of the Computer Science Department of the University of Pretoria. The document also serves to give the client and developers a clear description and elaboration of the system to be implemented in its totality.

\section{Vision}

The project aims to provide an online space which will be integrated into the CS website, where students, teaching assistants, and lecturers can engage in activities
related to learning the content of our module.The system will also apply game concepts to motivate students to increase the quality of their participation and consequently experience deeper learning of the course content.

\section{Background and System Description}

This project is due to the Computer Science department of the University of Pretoria having problems with the currently available tools for discussion forums, the following
problems are hampering positive engagement of both teaching staff and students: Unorganised content, user inexperience and  low levels of excitement.
\newline
The System intends to create an online discussion forum that has automated feedback on common mistakes, game-like presention as well as automated feedback. The system also provides the COS 301 students with the opportunity to learn about the procedures used for creating, designing and developing projects for businesses, while also providing the University with a potentially new system that may, be released as an opensource project, that could possibly be implemented worldwide.
\subsection{Related project}
The project is a face lift to the existing dicussion forum of the Department of Computer Sciences and aims to improve the existing one by bringing new features that help Students to be more involved in dicussing certian modules.
\subsection{System Environment}
The system will interract with LDAP ,which will handle credentials avoiding the need for a database

\section{The Stakeholders}

\subsection{The Client}
The Client is Ms Vreda Pieterse at the Department of Computer Science.
\subsection{The customers}
The Customers are Students of the Computer science who are enrolled in the modules and lectures in the department
\subsection{Maintenance Users}
The system will be assigned admimstartors from the Computer Science department and they will ensure its maintenace




\newpage
\section{Functional Requirements}
Temporary words
\subsection{Scope and Limitations/Exclusions}
Temporary words
\subsection{Use case Prioritization}
Temporary words
\subsection{Use case/Services Contracts}
Temporary words
\subsection{Required Functionality}
Temporary words
\subsection{Process Specifications}
Temporary words
\subsection{Domain Model}
Temporary words

\newpage
\section{Temporary Space}
\subsection{Kyhle}
Use diagrams for points 4,5,and 6
\subsubsection{Point 1:}
\begin{enumerate}
\item 
\textbf{Scope:}
Users should be able to create, read, update and delete posts.
\newline
\textbf{Limitations/exclusions:} Not all users should be able to use all the functions. Some users may even CRUD other users posts.

\item 
\textbf{Use case Prioritization:} Critical

\item 
\textbf{Use case/Service Contracts:} 

\textbf{Pre-Conditions: }
\begin{itemize}
\item Buzz space must exist.
\item User must be connected to the buzz system.

\item \textit{Create: }
	\begin{itemize}
    	\item Must have necessary permission to create posts.
    	\item Must be registered on the buzz system.
  	\end{itemize}

\item \textit{Read: }
	\begin{itemize}
	\item Post must exist.
	\end{itemize}
	
\item \textit{Update: }
	\begin{itemize}
	\item Post must exist.
	\item Must either be owner of the post, or have necessary 		permissions to update the post.
	\end{itemize}

\item \textit{Delete: }
	\begin{itemize}
	\item Post must exist.
	\item Must either be owner of the post, or have necessary permissions to delete the post
	\end{itemize}
\end{itemize}
 

\textbf{Post-Conditions: }
\begin{itemize}

\item \textit{Create: }
	\begin{itemize}
	\item Post will have been created.
	\item Post may not have been created, due to some error.
	\end{itemize}
\item \textit{Read: }
	\begin{itemize}
	\item If logged in, post will be marked as read for the specific user.
	\end{itemize}
\item \textit{Update: }
	\begin{itemize}
	\item Post will be updated if user has required permissions.
	\item Post may not have been updated due to some error.
	\end{itemize}
\item \textit{Delete: }
	\begin{itemize}
	\item Post will be marked as deleted, and thus removed from the discussion board.
	\item Post is not actually removed from the server, it is however hidden from all users.
	\item Post may not have been deleted due to some error. 
	\end{itemize}

\end{itemize}
\end{enumerate}

\subsubsection{Point 2:} 
\begin{enumerate}
\item 
\textbf{Scope:}
The system must keep track of who has read what, and highlight unread messages for each user.
\newline
\textbf{Limitations/exclusions:} 

\item 
\textbf{Use case Prioritization:} Critical

\item 
\textbf{Use case/Service Contracts:} 
\newline
\textbf{Pre-Conditions: }
\begin{itemize}
\item Buzz space must exist.
\item User must be registered to the buzz system.
\item User must be logged into the system while viewing the post.
\end{itemize}
 

\textbf{Post-Conditions: }
\begin{itemize}
\item Post is marked as read.
\item Post remains unmarked due to some error.
\end{itemize}
\end{enumerate}


\subsubsection{Point 3:} 
\begin{enumerate}
\item 
\textbf{Scope:}
This deals with the restriction of posting messages.
\newline
\textbf{Limitations/exclusions:} 
Message length should be restricted. Content type should also be restricted based on level and status of the user posting the message.

\item 
\textbf{Use case Prioritization:} Critical

\item 
\textbf{Use case/Service Contracts:} 
\newline
\textbf{Pre-Conditions: }
\begin{itemize}
\item Buzz space must exist.
\item User must be registered to the buzz system.
\item User must have necessary permissions to create posts of a certain length or content type.
\item Content type and message length must be established by the creator of that specific buzz (configurable).
\end{itemize}
 

\textbf{Post-Conditions: }
\begin{itemize}
\item Post is created.
\item Post may not have been created due to some error.
\end{itemize}
\end{enumerate}

\newpage

\subsection{Andrew}
<<<<<<< HEAD
\subsubsection{Point 1:}
\begin{enumerate}
\item
•	Restrict users to post on specified levels based on their status of the user posting the message.
\newline
\textbf{Use case Prioritization:} Important
\newline
Use case/Services contracts
\newline
Pre-Conditions:
\newline
Creator of the forum will not have any restrictions and can create new threads.
\newline
Post-Conditions:
Non-registered user:
User will not be able to post comment create new threads without acquiring an account.
\newline
Low level user:
When a user is on level x users may only post y number of topic or comments on other users posts per day. New posts can only be posted in pre-existing threads made by higher level users and no new threads may be made by low level users. 
\newline
Medium level user:
User may add and create any number of posts per day but are limited to the z number of threads that they can create per day.
\newline
High level user:
High level users have no posting restrictions. They made add any number of threads and are not limited to any number of posts per day.
 \newline
 All values are configurable by the Creator.
 \newline
 \end{enumerate}
 
\subsubsection{Point 2:}
•	Allow staff to manage content i.e. summaries, close or hide threads and move things around.
\newline
\textbf{Use case Prioritization:} Important
	

Use case/Services contracts
\newline
Pre-Conditions:
\newline
Users may not edit, move or change another user’s posts threads or comments that are on the same or on a higher level that they are.
\newline
Post-Conditions:
\newline
Staff or high level users:
User may move threads to their relevant categories. Remove unused or unimportant threads. Lock or close threads so users cannot post within them if the topic has been already been answered. Post may be edited to fix error or to remove irrelevant data.
\newline
Medium level user:
User may move posts to the relevant threads but may not edit change or update user’s posts.
\newline
Low level and non-registered users:
User may not edit, move or change another user’s posts threads or comments.
\newline
All values are configurable by the Creator.
\newline
 
\subsubsection{Point 3:}
•	Provide functionality to support semi-automatic creation of thread summaries.
\newline
\textbf{Use case Prioritization:} Nice-To-Have
Use case/Services contracts
\newline
Pre-Conditions:
\newline
Medium, High and Creator may state whether a certain thread may or may not be summarized.
\newline
Post-Conditions:
When a Thread has been marked for summary then all posts which have been highly rated and important would be moved closer to the top and unrelated or unnecessary posts would be move further away from the top to the bottom with all linked comments. High level users may override whether the summarized posts are relevant and can manually move the posts to where they best fit.
\newline
All levels of users are able to rate posts and comments on how important they see the post.
\newline
High level users would get the highest x points per rating.
Medium level users would get the highest y points per rating.
Low level users would get the highest z points per rating.
Non registered users cannot rate posts.
\newline
All values are configurable by the Creator.
\newline

=======
\begin{enumerate}
\item 
\item 
\item 
\item 
\item 
\item 
\end{enumerate}

\newpage

\subsection{Sifiso}
\subsubsection{Point 1:} 
\begin{enumerate}
\item 
\textbf{Scope:}
This deals with social tagging,broad folksnomy type is used in this social tagging
\newline
\textbf{Limitations/exclusions:} 
Not all users will be able to tag a buzz space,Users with higher privilagies and lectures will be able to social tag buzz space

\item 
\textbf{Use case Prioritization:} nice to have

\item 
\textbf{Use case/Service Contracts:} 
\newline
\textbf{Pre-Conditions: }
\begin{itemize}
\item User must be registered to the buzz system.
\item User must have necessary perivilages.
\item Buzz space must have a rating from users to be tagged
\end{itemize}
 

\textbf{Post-Conditions: }
\begin{itemize}
\item Buzz space tagged with a keyword.
\item Buzz space avilable at tag box for fast access.
\end{itemize}
\end{enumerate}

\subsubsection{Point 2:} 
\begin{enumerate}
\item 
\textbf{Scope:}
This deals with  self-organisation based on social tagging and allow the user to view according to the base structure, owns structure or public structure.
\newline
\textbf{Limitations/exclusions:} 
Users with higher privilagies will be able to organise view to thier own structure.
\item 
\textbf{Use case Prioritization:} nice to have

\item 
\textbf{Use case/Service Contracts:} 
\newline
\textbf{Pre-Conditions: }
\begin{itemize}
\item User must be registered to the buzz system.
\item User must have necessary perivilages.
\item Buzz space must have a rating from users.
\item Social tagging must be applied to other buzz space
\end{itemize}
 

\textbf{Post-Conditions: }
\begin{itemize}
\item Tagged buzz space with higher rating from users will be organised to be in base structure
\item Most accesed tagged buzz space will be included in public structure.
\item Users with buzz space can organise thier own structure .
\end{itemize}
\end{enumerate}

\subsubsection{Point 3:} 
\begin{enumerate}
\item 
\textbf{Scope:}
This deals with an addition of a read later section, which saves buzz space with long comments for a user to read later when logged in and remind the user every time when logged.
\newline
\textbf{Limitations/exclusions:} 
All users will have this feature by default
\item 
\textbf{Use case Prioritization:} nice to have

\item 
\textbf{Use case/Service Contracts:} 
\newline
\textbf{Pre-Conditions: }
\begin{itemize}
\item User must be registered to the buzz system.
\end{itemize}
 

\textbf{Post-Conditions: }
\begin{itemize}
\item Read later section will be created and added on the side of the users portal
\item Buzz space reference must be saved in a read later section on the user's portal is the user clicked a buzz space to read later section 
\end{itemize}
\end{enumerate}

\subsubsection{Point 4:} 
\begin{enumerate}
\item 
\textbf{Scope:}
This deals with buzz space tagging for least privilege users and will help social tagging by providing trending buzz space
\newline
\textbf{Limitations/exclusions:} 
All users will have this feature by default
\item 
\textbf{Use case Prioritization:} nice to have

\item 
\textbf{Use case/Service Contracts:} 
\newline
\textbf{Pre-Conditions: }
\begin{itemize}
\item User must be registered to the buzz system.
\item User must be allowed to the buzz space.

\end{itemize}
 

\textbf{Post-Conditions: }
\begin{itemize}
\item Menu will be provided to the user with Glyphicons  to describe the buzz space thread for the user to click on.
\item After user click a certain Glyphicons ,thread will be rated according to the Glyphicon
\end{itemize}
\end{enumerate}



\newpage

\subsection{Maret}

\subsubsection{Point 1:}
\begin{enumerate}
\item 
\textbf{Scope}
 The system must send template messages automatically to individual users or specified groups, like a welcomming message or a notification message.
\newline
\textbf{Limitations/exclusions:}
\item
\textbf{Use case Prioritization:} Nice-to-have
\textbf{Use case/Service Contracts:} 
\newline
\textbf{Pre-Conditions: }
\begin{itemize}
\item The user must be registered to the buzz system.
\item The message template should exist.
\item The system must be able to select a cetain group based on specific information, to send the group message to.
\end{itemize}
 \textbf{Post-Conditions: }
\begin{itemize}
\item The user must be alerted of the message.
\item The user must not be albe to reply to the message.
\item The user must be able to delete the messages.
\item The user must not be able to see what other users have received the same message via group messaging.
\end{itemize}
\end{enumerate}

\subsubsection{Point 2:}
\begin{enumerate}
\item 
\textbf{Scope}
The system must automatically change the status of a user based on his participation.
\newline
\textbf{Limitations/exclusions:}
\item
\textbf{Use case Prioritization:} Important
\textbf{Use case/Service Contracts:} 
\newline
\textbf{Pre-Conditions: }
\begin{itemize}
\item The user must be registered to the buzz system.
\item The user must be logged in for his status to be affected by his participation.
\item The users current status will be updated, so he must have a current status.
\end{itemize}
 \textbf{Post-Conditions: }
\begin{itemize}
\item The users privileges change when his status change.
\item The user must be able to view his status.
\item The users stutus is public.
\end{itemize}
\end{enumerate}


\subsection{Matthew}
\subsubsection{Point 1:}
\begin{enumerate}
\item 
\textbf{Scope:}
Statistical information created from evaluation to capture average mark of every student within a time frame. Visual reporting of a participants evaluation in correlation to the average of the evaluation of all users or certain groups of users for gamification concept.
\newline
\textbf{Limitations/exclusions:} : Not all users may see statistical information of users. Only higher level users may view statistical information and visual reporting of users.

\item 
\textbf{Use case Prioritization:} Important.

\item 
\textbf{Use case/Service Contracts:} 

\textbf{Pre-Conditions: }
\begin{itemize}
\item User must be connected to buzz.

\item User must be part of discussion or module to be evaluated.

\item User may not be of higher level of the users who monitor the statistical information.


\end{itemize}
 

\textbf{Post-Conditions: }
\begin{itemize}

\item Results and statistical information may not be altered.
\item No more statistical information may be formulated once time period passes.
\item All users may see there statistical information.
\item Statistical information will be represented in a visual representation from that time period.



\end{itemize}
\end{enumerate}

\subsubsection{Point 2:} 
\begin{enumerate}
\item 
\textbf{Scope:}
Improved post editor for example text formatting and automatic pretty printing of code in posts.
\newline
\textbf{Limitations/exclusions:} 

Not all users will be able to use these options. Has to be earned. Certain options are available for all users by default.

\item 
\textbf{Use case Prioritization:} Nice-To-Have.

\item 
\textbf{Use case/Service Contracts:} 
\newline
\textbf{Pre-Conditions: }
\begin{itemize}
\item User must be connected to buzz system.
\item User must be of a high enough level.
\item Users must be using the post editor.
\end{itemize}
 

\textbf{Post-Conditions: }
\begin{itemize}

\item Message will be formatted accordingly and marked up (e.g. font colouring,emoticons,various fonts)to how the user wanted. 
\item User must send message to be viewable by other users.
\end{itemize}
\end{enumerate}

\subsection{Sphelele}
\subsubsection{Point 1: }
\begin{enumerate}
\item \textbf{Scope: }
The website will allow users to search the website for topics and buzz spaces, and then filter those search results by various categories. \newline \newline
	  \textbf{Limitation/Exclusions: }
Users will be limited to 4 filter categories, namely: topic, date posted/last updated, buzz space name and rating.
\item \textbf{Use case Prioritization: } Critical
\item \textbf{Use case/Service Contracts: } \newline \newline
	  \textbf{Main Scenario: }
	  \begin{itemize}
	  \item User enters search term into search bar
	  \item Website returns partial and complete matches to the users search query
	  \item Filter options appear along with search results
	  \item User selects filter(s) and search results are refined according to filter.
	  \item User clicks on desired search result or searches with different search query and process repeats.
	  \end{itemize}
	  \item
	  \item
	  \item
	  \end{enumerate}
	  
	  \subsubsection{Point 2:}
	  \begin{enumerate}
	  
	  
	  \item \textbf{Scope:} The website will allow users to evaluate/vote for posts on the website. \newline \newline
	  \textbf{Limitations/Exclusions:} Only users that are logged in to the system and have the privilege rights to evaluate/vote on posts will be allowed to evaluate/vote on posts. Higher Privileged users' votes will push posts higher (or lower) than lower privileged users. \newline \newline
	  
	  \item \textbf{Use case Prioritization: } Critical \newline \newline
	  \item \textbf{Use Case/Service Contract: } \newline \newline
	  \textbf{Pre-Conditions}
	  \begin{itemize}
	  
	  \item User privilege level allows user to vote for post
	  \item post must exist and not be removed
	  \end{itemize}
	  \textbf{Post Conditions}
	  \begin{itemize}
	  \item Post rating is updated.
	  \item post moves up/down accordingly
	  \end{itemize}
	  \item
	  \item
	  \item
	  \end{enumerate}

\newpage

\subsection{name}
\begin{enumerate}
\item 
\item 
\item 
\item 
\item 
\item 
\end{enumerate}

\newpage

\subsection{name}
\begin{enumerate}
\item 
\item 
\item 
\item 
\item 
\item 
\end{enumerate}

\newpage

\end{document}
